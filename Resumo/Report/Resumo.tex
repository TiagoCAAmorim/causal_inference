\documentclass[final,5p]{elsarticle}

% \documentclass[preprint,12pt]{elsarticle}

%% Use the option review to obtain double line spacing
%% \documentclass[authoryear,preprint,review,12pt]{elsarticle}

%% Use the options 1p,twocolumn; 3p; 3p,twocolumn; 5p; or 5p,twocolumn
%% for a journal layout:
% \documentclass[final,1p,times]{elsarticle}
%% \documentclass[final,1p,times,twocolumn]{elsarticle}
% \documentclass[final,3p,times]{elsarticle}
%% \documentclass[final,3p,times,twocolumn]{elsarticle}
% \documentclass[final,5p,times]{elsarticle}
%% \documentclass[final,5p,times,twocolumn]{elsarticle}
\usepackage[portuguese]{babel}

%% For including figures, graphicx.sty has been loaded in
%% elsarticle.cls. If you prefer to use the old commands
%% please give \usepackage{epsfig}

%% The amssymb package provides various useful mathematical symbols
\usepackage{amssymb}
\usepackage{amsmath}
\usepackage{multirow}
\usepackage{tabularx}
\usepackage{booktabs}
\usepackage{tablefootnote}

\usepackage{pgfplots}
\pgfplotsset{compat=1.18}
\usepgfplotslibrary{statistics}
\usepackage{pgfplotstable}

\usepackage{placeins}
\usepackage{hyperref}
\numberwithin{equation}{section}

\usepackage{algorithm}
\usepackage[noEnd=true, indLines=true]{algpseudocodex}
\algrenewcommand\algorithmicrequire{\textbf{Entrada:}}
\algrenewcommand\algorithmicwhile{\textbf{Enquanto}}
\algrenewcommand\algorithmicrepeat{\textbf{Repete}}
\algrenewcommand\algorithmicuntil{\textbf{Até}}
\algrenewcommand\algorithmicif{\textbf{Se}}
\algrenewcommand\algorithmicthen{\textbf{então}}
\algrenewcommand\algorithmicelse{\textbf{Caso contrário}}
\algrenewcommand\algorithmicensure{\textbf{Objetivo:}}
\algrenewcommand\algorithmicreturn{\textbf{Retorna:}}
\algrenewcommand\algorithmicdo{\textbf{faça}}
\algrenewcommand\algorithmicforall{\textbf{Para todos}}
\algnewcommand{\LineComment}[1]{\State \(\triangleright\) \textcolor{black!50}{\emph{#1}}}

\newcommand*{\squareb}{\textcolor{black}{\rule{0.5em}{0.5em}}}
\newcommand*{\squareg}{\textcolor{gray}{\rule{0.5em}{0.5em}}}

\graphicspath{ {./png/} }

% \usepackage[fleqn]{nccmath}
% \usepackage{multicol}


%=========== Gloabal Tikz settings
% \pgfplotsset{compat=newest}
% \usetikzlibrary{math}
% \pgfplotsset{
%     height = 10cm,
%     width = 10cm,
%     tick pos = left,
%     legend style={at={(0.98,0.30)}, anchor=east},
%     legend cell align=left,
%     }
%  \pgfkeys{
%     /pgf/number format/.cd,
%     fixed,
%     precision = 1,
%     set thousands separator = {}
% }

%% The amsthm package provides extended theorem environments
%% \usepackage{amsthm}

%% The lineno packages adds line numbers. Start line numbering with
%% \begin{linenumbers}, end it with \end{linenumbers}. Or switch it on
%% for the whole article with \linenumbers.
%% \usepackage{lineno}

\usepackage{listings}
\usepackage{xcolor}

\definecolor{codegreen}{rgb}{0,0.6,0}
\definecolor{codegray}{rgb}{0.5,0.5,0.5}
\definecolor{codepurple}{rgb}{0.58,0,0.82}
\definecolor{backcolour}{rgb}{0.98,0.98,0.98}

\lstdefinestyle{mystyle}{
    backgroundcolor=\color{backcolour},
    commentstyle=\color{codegreen},
    keywordstyle=\color{magenta},
    numberstyle=\tiny\color{codegray},
    stringstyle=\color{codepurple},
    basicstyle=\ttfamily\footnotesize,
    breakatwhitespace=false,
    breaklines=true,
    captionpos=b,
    keepspaces=true,
    numbers=left,
    numbersep=5pt,
    showspaces=false,
    showstringspaces=false,
    showtabs=false,
    tabsize=2
}

\lstset{style=mystyle}

% \journal{Nuclear Physics B}

\begin{document}

\begin{frontmatter}

%% Title, authors and addresses

%% use the tnoteref command within \title for footnotes;
%% use the tnotetext command for theassociated footnote;
%% use the fnref command within \author or \address for footnotes;
%% use the fntext command for theassociated footnote;
%% use the corref command within \author for corresponding author footnotes;
%% use the cortext command for theassociated footnote;
%% use the ead command for the email address,
%% and the form \ead[url] for the home page:
%% \title{Title\tnoteref{label1}}
%% \tnotetext[label1]{}
%% \author{Name\corref{cor1}\fnref{label2}}
%% \ead{email address}
%% \ead[url]{home page}
%% \fntext[label2]{}
%% \cortext[cor1]{}
%% \affiliation{organization={},
%%             addressline={},
%%             city={},
%%             postcode={},
%%             state={},
%%             country={}}
%% \fntext[label3]{}

\title{Tópicos de Inferência Causal\tnoteref{label_title}}
\tnotetext[label_title]{Relatório como parte dos requisitos da disciplina MI628: Inferência Causal.}

%% use optional labels to link authors explicitly to addresses:
%% \author[label1,label2]{}
%% \affiliation[label1]{organization={},
%%             addressline={},
%%             city={},
%%             postcode={},
%%             state={},
%%             country={}}
%%
%% \affiliation[label2]{organization={},
%%             addressline={},
%%             city={},
%%             postcode={},
%%             state={},
%%             country={}}

\author[label1]{Tiago C A Amorim (RA: 100675)}
\affiliation[label1]{organization={Doutorando no Departamento de Engenharia de Petróleo da Faculdade de Engenharia Mecânica, UNICAMP},
            city={Campinas},
            state={SP},
            country={Brasil}}


% \begin{abstract}

%     xxxxxxx

% \end{abstract}


%%Graphical abstract
% \begin{graphicalabstract}
%\includegraphics{grabs}
% \end{graphicalabstract}

%%Research highlights
% \begin{highlights}
% \item Research highlight 1
% \item Research highlight 2
% \end{highlights}

\begin{keyword}
    Inferência Causal \sep Controle Negativo
%% keywords here, in the form: keyword \sep keyword

%% PACS codes here, in the form: \PACS code \sep code

%% MSC codes here, in the form: \MSC code \sep code
%% or \MSC[2008] code \sep code (2000 is the default)

\end{keyword}

\end{frontmatter}

%% main text
\section{Introdução}

    Este relatório é um resumo de dois artigos na área de inferência causal. O primeiro artigo tem um enfoque mais gerenalista, e aborda diversos assuntos da área \cite{imbens2024causal}. O segundo artigo foca em controles negativos\footnote{Tradução livre de \emph{negative controls}.} \cite{lipsitch2010negative}. Na medida do possível este resumo buscará seguir a notação utilizada ao longo do curso de inferência causal, que segue o livro de Peng Ding \cite{ding2023first}.

\section{\textit{Causal Inference in the Social Sciences}}

    Este artigo discute diversos assuntos ligados à avaliação de inferência causal. Será dado maior enfoque às novas técnicas e discussões trazidas pelo autor no lugar de descrever as metodologias já vistas durante a disciplina MI628: Inferência Causal.

    \subsection{Estudos Experimentais Randomizados}

    O autor cita que estudos experimentais randomizados surgem nos anos de 1920, com Fisher e Neyman. Apesar destas ferramentas ainda serem utilizadas, novas ferramentas foram desenvolvidas. O autor cita que em geral o objetivo de um estudo de inferência causal está mais ligado ao resultado médio do tratamento (visão \textit{neymaniana}) e sua magnitude.

    A técnica de \textbf{experimentos adaptativos}\footnote{Tradução livre de \textit{adaptative experiments}.} foi desenvolvida para tratar de cenários em que existe uma resposta rápida ao tratamento, existem diferentes tratamentos disponíveis, a atribuição das unidades é sequencial e cujo o objetivo é descobrir o \textit{melhor} tratamento, e não a magnitude do seu efeito. O autor dá como exemplo a avaliação de qual é a melhor progaganda a ser veiculada online.

    Em um estudo tradicional as unidades seriam alocadas de forma igual entre os diferentes tratamentos. São apresentadas duas técnicas de experimentos adaptativos que recalculam a probabilidade de atribuição das unidades aos tratamentos, em função dos resultados alcançados até determinado momento. Uma técnica usa a probabilidade posterior de determinado tratamento ser o melhor como peso na sua atribuição. A segunda técnica usa o limite superior dos intervalos de confiança do sucesso de um tratamento: \textbf{UCB} (\textit{upper confidence bounds}). A próxima unidade é atribuída o tratamento com o maior UCB. Em ambas abordagens é preciso cuidar do balanço entre exploração e explotação.

    Problemas citados pelo autor: viés de inferência ao usar média dos resultados como estimador da esperança dos resultados, efeito de covariáveis na probabilidade de atribuição, respostas que variam com o tempo (estocacidade ou sazonalidade).

    Uma outra situação que precisa de tratamento especial é quando as condições de não-interferência e consistência (valor estável de tratamento da unidade ou SUTVA\footnote{Do inglês: \textit{Stable Unit Treatment Value Assumption}.}) não são atendidas. Quando existe a possibilidade de transbordamento ou interferência entre as unidades, a estratégia é definida em função do tipo de transbordamento. Uma estratégia é particionar a população em subgrupos, de forma a limitar os efeitos de transbordamento ao subgrupo. Outra estratégia é com o uso de grafos bipartidos e clusterização aleatória. O último exemplo dado pelo autor é quando duas ou mais populações estão associadas a tratamentos, e as variações de atribuição dos tratamentos nos pares (ou triplas, ...) de indivíduos ajuda a estimar os efeitos de transbordamento.

    \subsection{Estudos Observacionais sob Ignorabilidade}

    Em estudos observacionais, assumir que a atribuição de tratamento é basicamente independente (condição de \textbf{ignorabilidade} forte: $Z_i \perp (Y_i(0),Y_i(1)) \mid X_i$) leva à possibilidade de analisar os dados como se fossem de um experimento randomizado. Além disto é comum assumir que a probabilidade de atribuição (\textbf{\textit{propensity score}}) é limitada longe de zero e um: $e(x) = pr(Z_i=1 \mid X_i=x)\, \epsilon \, (\delta, 1 - \delta), \; \text{com} \; \delta > 0$ (\textbf{sobreposição}).

    A estimativa do efeito médio do tratamento sob ignorabilidade pode ser feita com diferentes abordagens (com comentários do autor):
    \begin{itemize}
        \item \textit{Matching}: Estimador intuitivo, mas necessita de grande número de \textit{matches} e sofre quando existe um número grande de covariáveis.
        \item Regressão dos estimadores: Pode ser difícil de construir bons estimadores quando a dimensão das covariáveis é muito grande.
        \item Estimadores com o \textit{propensity score}: Pode ser utilizado como uma covariável escalar junto com \textit{matching} ou regressão, ou usar seu inverso como peso na estimativa do efeito causal médio.
        \item Estimadores duplamente robustos: É a classe de estimadores atualmente sugerida na literatura, pois tem boas propriedades mesmo que os estimadores tenham baixa convergência.
    \end{itemize}

    Violações de sobreposição podem ser contornadas ignorando amostras que estão muito próximas de zero ou um, ou usando $h(x) = e(x)(1-e(x))$ como peso no estimador do efeito causal médio.

    A estimativa do efeito causal médio condicionado em covariáveis ($\tau(x) = E[Y_i(1)-Y_i(0) \mid X_i=x]$) é uma área de estudo mais recente. Segundo o autor nesta área existem bons resultados com métodos de aprendizado de máquina, como florestas de regressão e florestas aleatórias.

    Por último o autor discorre sobre linhas de pesquisa que buscaram estimar o efeito de assumir ignorabilidade em aplicações onde esta não pode ser assumida exatamente. São citadas três abordagens:

    \begin{itemize}
        \item Assumir que não existe ignorabilidade e estimar limites para o efeito médio do tratamento. Esta abordagem pode levar a limites muito grandes.
        \item Assumir ignorabilidade, mas condicional a uma covariável não observada, que tem limitada associação com os resultados potenciais e a atribuição de tratamento.
        \item x
    \end{itemize}

    \subsection{Estudos Observacionais sem Ignorabilidade}



    \subsection{Combinação de Estudos Experimentais e Observacionais}



\section{\textit{Negative Controls: A Tool for Detecting Confounding and Bias in Observational Studies}}

    É apresentado o conceito de controle negativo. Este mecanismo é utilizado em estudos de biologia experimental, e os autores argumentam que pode ser adapatado para experimentos observacionais e de outras áreas. Associações não-causais podem ser causadas por erro de medição, confusão e viés de seleção de indivíduos para a análise. O uso de controles negativos ajuda a verificar o risco de interpretação não-causal como causal.

    A ideia principal do contole negativo é repetir o experimento que atestou efeito causal, mas de forma que o resultado esperado seja a hipótese nula, i.e., sem efeito causal. Vários controles negativos podem ser necessários, em função dos possíveis efeitos não-causais (variáveis de confusão não medidas). A observação de efeito causal no controle negativo é indicativo de que variáveis não medidas estão influenciando os resultados de interesse. Mecanismos de construção de experimentos de controle negativo citados pelos autores:

    \begin{itemize}
        \item Retirar um dos elementos essenciais ao mecanismo em avaliação: Testar hipótese de viés de disponibilidade com perguntas \emph{placebo}, que já se sabe não ter relação com o mecanismo avaliado, em um questionário sobre alguma doença.
        \item Neutralizar o efeito do mecanismo: Avaliar saúde de pacientes que recebem doses ineficazes de algum medicamento.
        \item Testar com condições em que o mecanismo avaliado não funciona: Índice de hospitalizações por quebra de ossos em função de tomar ou não a vacina da gripe.
    \end{itemize}

    Em resumo, o propósito do controle negativo é de testar condições em que o mecanismo avaliado não está ativo, mas que podem sofrer dos mesmos vieses (medidos e não-medidos) do experimento original. Seguindo a Figura \ref{fig:diagrama_confusao}, uma resposta de controle negativo é uma resposta (\textbf{N}) que é influenciada pelos mesmos parâmetros que influenciam o tratamento (\textbf{Z}) e a resposta de interesse (\textbf{Y}). Se as variáveis de confusão não medidas (\textbf{U}) forem as mesmas para os pares \textbf{Z}+\textbf{Y} e \textbf{Z}+\textbf{N}, então \textbf{Y} e \textbf{N} são ditos \textit{U-comparáveis}.

    Assumindo que esta nova resposta (\textbf{N}) não é influenciada pelo tratamento (\textbf{Z}), e que a resposta de interesse (\textbf{Y}) e a nova resposta (\textbf{N}) são \textit{U-comparáveis}, se não é observado efeito causal de \textbf{Z} em \textbf{N}, então podemos afirmar que \textbf{provavelmente} um efeito causal observado de \textbf{Z} em \textbf{Y} não é resultado de um viés desconhecido. Segundo os autores, na prática não é possível tecer uma afirmação estrita sobre a causalidade de \textbf{Z} em \textbf{Y} porque em geral \textbf{Y} e \textbf{N} são apenas aproximadamente \textit{U-comparáveis}.

    \begin{figure}[hbt!]
        % \includegraphics[width=0.95\columnwidth]{A_LabelCount.png}
        \centering
        \begin{tikzpicture}[node distance={15mm}, thick, main/.style = {draw, circle}]
            \node[main] (1) {$Z$};
            \node[main] (2) [above of=1] {$X$};
            \node[main] (3) [right of=1] {$Y$};
            \node[main] (4) [gray, left of=2] {$U$};
            \node[main] (5) [red, above of=4] {$N$};
            \draw[->] (2) -- (1);
            \draw[->] (2) -- (3);
            \draw[->] (1) -- (3);
            \draw (4)[gray, ->] to [out=270,in=225,looseness=1.5] (3);
            \draw[gray, ->] (4) -- (1);
            \draw[gray, dashed, <->] (2) -- node[midway, above right, sloped, pos=1] {?} (4);
            \draw[red, ->] (2) -- (5);
            \draw[red, ->] (4) -- (5);
        \end{tikzpicture}
        \caption{Diagrama de causalidade: efeito do tratamento \textbf{Z} no resultado \textbf{Y}, com as variáveis de confusão \textbf{X} (medida) e \textbf{U} (não medida). A linha tracejada entre \textbf{X} e \textbf{U} indica que um pode influenciar o outro. Neste exemplo os resultados \textbf{N} e \textbf{Y} são ditos U-comparáveis. (Adapatado de \cite{lipsitch2010negative})}
        \label{fig:diagrama_confusao}
    \end{figure}

    A mesma lógica pode ser aplicada para encontrar um tratamento de controle negativo (\textbf{B}). Se as variáveis de confusão não medidas (\textbf{U}) forem as mesmas para os pares \textbf{Z}+\textbf{Y} e \textbf{B}+\textbf{Y}, então \textbf{Z} e \textbf{B} são ditos \textit{U-comparáveis}. Novamente, se não existe causalidade de \textbf{B} em \textbf{Y}, então uma causalidade observada de \textbf{Z} em \textbf{Y} provavelmente não é efeito de um viés desconhecido.

    Os autores argumentam que é uma boa prática usar vários controles negativos para ajudar na validação de um estudo que apontou causalidade, e que conhecimento do assunto objeto do estudo é crucial para fazer boas escolhas de controles negativos.

\section{Discussão}


% \appendix


%% If you have bibdatabase file and want bibtex to generate the
%% bibitems, please use
%%

\bibliographystyle{elsarticle-num}
\bibliography{refs}

%% else use the following coding to input the bibitems directly in the
%% TeX file.

% \begin{thebibliography}{00}

%% \bibitem{label}
%% Text of bibliographic item

% \bibitem{}

% \end{thebibliography}


\end{document}
\endinput
